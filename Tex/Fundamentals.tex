% Header
%%%%%%%%%%%%%%%%%%%%%%%%%%%%%%%%%%%%%%%%%%%%%%%%%%%%%%%%%%%%%%%%%%%%%%%%%%%%%%%%%%%

\chapter{Fundamentals}
\label{fundamentals}
\thispagestyle{empty}

%%%%%%%%%%%%%%%%%%%%%%%%%%%%%%%%%%%%%%%%%%%%%%%%%%%%%%%%%%%%%%%%%%%%%%%%%%%%%%%%%%%


This chapter includes the most important fundamentals about the cardiac anatomy and physiology, as well as the necessary technical background for understanding the methods developed during the research for this master thesis. The first section covers the human physiology.





% First Section: Physiological Fundamentals
%%%%%%%%%%%%%%%%%%%%%%%%%%%%%%%%%%%%%%%%%%%%%%%%%%%%%%%%%%%%%%%%%%%%%%%%%%%%%%%%%%

\section{Physiological Fundamentals}
\label{physiologicalFundamentals}
Looking at a human heart from an engineer's perspective, it can be described as a highly optimized technical systems. The heart is a strong pump with pipes and valves and it is driven by electric excitation. Just like an artificial system it depends on some kind of control unit, which is able to adapt to different conditions and disturbances. Although the human heart is a very robust and durable system, its efficiency and functionality declines with time. On top the components of this complex system might also fail, if they are exposed to too much stress, again just like in an engineered system. 
% In order to detect and possibly solve these failures, the Electrocardiogram (ECG) represents a powerful tool. It is measured using electrodes, which record the electrical excitation on the skin.
\newline
This section describes the most important components of this complex object. It is not intended to give a detailed discourse of the anatomy and physiology yet the section introduces the necessary facts to understand the analysis methods, developed during the research for this thesis. The topics, which will be introduced, include the heart itself and the Autonomic Nervous System (ANS). The respiratory system and its influence on the functionality of the heart is furthermore described. Subsequently an explanation of the Electrocardiogram (ECG) and its interpretation can be found in the last part of this section.

% Anatomy of the Human Heart
% -----------------------------------------------------------------------------------------------------------------------------------------------------------------------------------------------------------------------
\subsection{The Anatomy of the Human Heart}
\label{anatomy}
% -----------------------------------------------------------------------------------------------------------------------------------------------------------------------------------------------------------------------



% Electrophysiology of the Human Heart
% -----------------------------------------------------------------------------------------------------------------------------------------------------------------------------------------------------------------------
\subsection{The Electrophysiology of the Human Heart}
\label{physiology}
% -----------------------------------------------------------------------------------------------------------------------------------------------------------------------------------------------------------------------



% The Autonomic Nervous System
% -----------------------------------------------------------------------------------------------------------------------------------------------------------------------------------------------------------------------
\subsection{The Autonomic Nervous System}
\label{ans}
% -----------------------------------------------------------------------------------------------------------------------------------------------------------------------------------------------------------------------



% The ECG
% -----------------------------------------------------------------------------------------------------------------------------------------------------------------------------------------------------------------------
\subsection{The Electrocardiogram (ECG)}
\label{ecg}
\paragraph{The Vectorcardiogram (VCG)}
\label{vcg}
\paragraph{The Respiratory Sinus Arrhythmia (RSA)}
\label{rsa}
% -----------------------------------------------------------------------------------------------------------------------------------------------------------------------------------------------------------------------

%%%%%%%%%%%%%%%%%%%%%%%%%%%%%%%%%%%%%%%%%%%%%%%%%%%%%%%%%%%%%%%%%%%%%%%%%%%%%%%%%%%





% Second Section: Mathematical Fundamentals
%%%%%%%%%%%%%%%%%%%%%%%%%%%%%%%%%%%%%%%%%%%%%%%%%%%%%%%%%%%%%%%%%%%%%%%%%%%%%%%%%%%

\section{Mathematical Fundamentals}
\label{mathematicalFundamentals}

%%%%%%%%%%%%%%%%%%%%%%%%%%%%%%%%%%%%%%%%%%%%%%%%%%%%%%%%%%%%%%%%%%%%%%%%%%%%%%%%%%%


