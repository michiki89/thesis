\chapter{Tabellen}
\label{Tabellen}
\thispagestyle{empty}

Wie erzeuge ich eine Tabelle ??

\vspace{5cm}

\begin{table}[h]
 \caption{Rechenvorteil der FFT gegen"uber der direkten FT}
 \begin{center}
 \begin{tabular}{|rrrr|}
 \hline
 & direkte FT & FFT & Rechenvorteil\\
 $N$ & $N^{2}$ & $N \log_{2} N$ & $\log_{2} N/N$ \\
 \hline
 \hline
   2&          4&        2&     50,0\%\\
   4&         16&        8&     50,0\%\\
   8&         64&       24&     37,5\%\\
  16&        256&       64&     25,0\%\\
  32&      1.024&      160&     15,6\%\\
  64&      4.096&      384&      9,4\%\\
 128&     16.384&      896&      5,5\%\\
 256&     65.536&    2.048&      3,1\%\\
 512&    262.144&    4.608&      1,8\%\\
1024&  1.048.576&   10.240&      1,0\%\\
2048&  4.194.304&   22.528&      0,5\%\\
4096& 16.777.216&   49.152&      0,3\%\\
8192& 67.108.864&  106.496&      0,2\%\\
 \hline
  \end{tabular}
 \end{center}
 \label{tablerechenvorteilfft}
\end{table}   

\newpage

Hier ist ein Beispiel f"ur eine colorierte Tabelle
\begin{table}[h!]
 \caption[Typische Verteilung der Ionen im Intra- und Extrazellul"arraum einer Muskelzelle.]{Typische Verteilung der Ionen im Intra- und Extrazellul"arraum einer Muskelzelle \cite{silbernagl01}. Durch den Konzentrationsunterschied entsteht f"ur jede Ionenart eine sogenannte Nernst-Spannung.}
\centering
\begin{tabular}{|>{\columncolor{hellblau}}c|c|c|c|}  
  \hline\rowcolor{hellblau}
  & intrazellul"are & exrazellul"are & Nernst-\\\rowcolor{hellblau}
   \, Ionenart\, &\, Konz. [$mmol/kg\,H_2O$]\, &\, Konz. [$mmol/kg\,H_2O$]\, &\, Spannung [mV]\, \\
  \hline \hline
  $K^{+}$&\cellcolor{hellgelb} 4,5 &\cellcolor{hellgelb} 160 &\cellcolor{hellgelb} -95,4\\
  $Na^{+}$&\cellcolor{hellgelb} 144 &\cellcolor{hellgelb} 7 &\cellcolor{hellgelb} 80,2\\
  $Ca^{2+}$&\cellcolor{hellgelb} 1,3 &\cellcolor{hellgelb} 0,00001-0,0001 &\cellcolor{hellgelb} 126,5-157,3\\
  $Cl^{-}$&\cellcolor{hellgelb} 114 &\cellcolor{hellgelb} 7 &\cellcolor{hellgelb} -74,5\\
  \hline
\end{tabular}
%  \label{ionenkonzentration}
 \vspace{2mm}
 \label{ionenkonzentration}
\end{table}


\newpage

In   diesem   Abschnitt   sollen lediglich   die   grundlegenden
Eigenschaften  der  Fouriertransformation kurz tabellarisch dargestellt werden
(Tabelle  \ref{tableeigenft}). Die  Beweise zu den Regeln und  alle  Eigenschaften
sind in \cite{gonzalez92bsp} zu finden.

\begin{table}[H]
 \begin{center}
  \begin{tabular}{|ccc|}
 \hline
 Ortsbereich &$\circ \hspace{-0.15cm} - \hspace{-0.15cm} \bullet$& Frequenzbereich\\\hline
 \hline
 Linearit"at &&  Linearit"at \\
 $k_{1}g(x)+k_{2}f(x)$ && $k_{1}G(u)+k_{2}F(u)$ \\\hline

 Symmetrie   &&  Symmetrie \\
 $F(x)$ && $f(-u)$ \\\hline

 Ortsskalierung && reziproke\\
 && Frequenzskalierung \\
 $f(kx)$ && $\frac{1}{|k|}F(\frac{u}{k})$\\\hline

 reziproke&&\\
 Ortsskalierung && Frequenzskalierung \\
 $\frac{1}{|k|}f(\frac{x}{k})$ && $F(ku)$\\\hline

 Ortsverschiebung && Phasenverschiebung \\
 $f(x-x_{0})$ && $F(u) e^{ -j 2 \pi u x_{0} }$\\\hline

 Modulation && Frequenzverschiebung \\
 $f(x) e^{ -j 2 \pi x u_{0} }$ && $F(u-u_{0})$\\\hline

 gerade Funktion && reelle Funktion \\
 $f_{g}(x)$ && $F_{g}(u)=R_{g}(u)$\\\hline

 ungerade Funktion && imagin"are Funktion \\
 $f_{u}(u)$ && $F_{u}(u)=jI_{u}(u)$\\\hline

 reelle Funktion && gerader Realteil,\\
 && ungerader Imagin"arteil \\
 $f(x)=f_{r}(u)$ && $F(u)=R_{g}(u)+j I_{u}(u)$\\\hline

 imagin"are Funktion && ungerader Realteil, \\
 && gerader Imagin"arteil \\
 $f(x)=j f_{i}(u)$ && $F(u)=R_{u}(u)+j I_{g}(u)$\\\hline

  \end{tabular}
 \end{center}
 \caption[Wichtige Eigenschaften der Fouriertransformation]{
Wichtige Eigenschaften der Fouriertransformation}
 \label{tableeigenft}
\end{table}
